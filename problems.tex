\documentclass{article}

\usepackage{array}
\usepackage{minted}


\title{Introduction to programming using python problems}
\date{Auguest 2021}


\begin{document}
\maketitle
\pagebreak
\tableofcontents
\pagebreak
\section*{Unit 1: Python basics}

\begin{enumerate}

  \item Assume \textbf{s} is a string of lower case characters. Write a program
    that counts up the number of vowels contained in the string \textbf{s}. Valid
    vowels are: \textbf{a', 'e', 'i', 'o', and 'u'}. For example,
    \textbf{if s = 'azcbobobegghakl'}, your program should print:\\
    \verb| Number of vowels: 5|

  \item Assume \textbf{s} is a string of lower case characters.Write a program \
    that prints the number of times the string \textbf{'bob'} occurs in \textbf{s}.
    For example, \textbf{if s = 'azcbobobegghakl'}, then your program should\
    print

    \verb|Number of times bob occurs is: 2|

  \item Assume \textbf{s} is a string of lower case characters. Write a program
    that prints the longest substring of \textbf{s} in which the letters occur
    in alphabetical order. For example, \textbf{if s = 'azcbobobegghakl'}, then
    your program should print

    \verb|Longest substring in alphabetical order is: beggh|

    In the case of ties, print the first substring. For example, \textbf{if s = 'abcbcd'},
    then your program should print

    \verb|Longest substring in alphabetical order is: abc|

    \textbf{Note}: This problem may be challenging. We encourage you to work smart. If
    you've spent more than a few hours on this problem, we suggest that you move
    on to a different part of the course. If you have time, come back to this
    problem after you've had a break and cleared your head.

\end{enumerate}

\pagebreak

\section*{Unit 2: Simple programs}
Each month, a credit card statement will come with the option for you to pay a
minimum amount of your charge, usually 2\% of the balance due. However, the credit
card company earns money by charging interest on the balance that you don't pay.
So even if you pay credit card payments on time, interest is still accruing on
the outstanding balance.

Say you've made a \$5,000 purchase on a credit card with an 18\% annual interest
rate and a 2\% minimum monthly payment rate. If you only pay the minimum monthly
amount for a year, how much is the remaining balance?

You can think about this in the following way.

At the beginning of month 0 (when the credit card statement arrives), assume you
owe an amount we will call $b_0$ (b for balance; subscript 0 to indicate this is the
balance at month 0).

Any payment you make during that month is deducted from the balance. Let's call
the payment you make in month 0,$p_0$. Thus, your \verb|unpaid balance| for month 0,$ub_0$, is
equal to $b_0-p_0$.

\[ub_0 = b_0 - p_0\]

At the beginning of month 1, the credit card company will charge you interest on
your unpaid balance. So if your annual interest rate is $r$, then at the beginning
of month 1, your new balance is your previous unpaid balance $ub_0$, \verb|plus| the interest
on this unpaid balance for the month. In algebra, this new balance would be

\[b_1 = ub_0 + r/12 * ub_0\]

In month 1, we will make another payment,$p_1$. That payment has to cover some of
the interest costs, so it does not completely go towards paying off the original
charge. The balance at the beginning of month 2,$b_2$, can be calculated by first
calculating the unpaid balance after paying $p_1$, then by adding the interest accrued:

\[ub_1 = b_1 - p_1\]
\[b_2 = ub_1 + r/12 * ub_1\]

If you choose just to pay off the minimum monthly payment each month, you will
see that the compound interest will dramatically reduce your ability to lower
your debt.

Let's look at an example. If you've got a \$5,000 balance on a credit card with
18\% annual interest rate, and the minimum monthly payment is 2\% of the current
balance, we would have the following repayment schedule if you only pay the
minimum payment each month:

\begin{table}[H]
  \centering

  \begin{tabular}{ | m{1cm} | m{3cm} | m{3cm} | m{3cm} | m{3cm} | }
    \hline
    Month & Balance & Minimum Payment & Unpaid Balance & Interest \\
    \hline
    0   & 5000.00 & 100(=5000 *.02) & 4900(=5000 -100) & 73.50(=.18/12.0 *4900)\\
    \hline
    1 & 4973.50(=4900 +73.50) & 99.47(=4973.50 *.02) & 4874.03(=4973.50 -99.47) & 73.11(=.18/12 *4874.03)\\
    \hline
    2 & 4947.14(=4874.03 +73.11) & 98.94(=4947.14 *.02) & 4848.20(=4947.14 -98.94) & 72.72(=.18/12 *4848.20)\\
    \hline
  \end{tabular}

\end{table}
  You can see that a lot of your payment is going to cover interest, and if you
  work this through month 12, you will see that after a year, you will have paid
  \$1165.63 and yet you will still owe \$4691.11 on what was originally a \$5000.00
  debt. Pretty depressing!

  \subsection*{Problem 1 - Paying Debt off in a Year }
  Write a program to calculate the credit card balance after one year if a
  person only pays the minimum monthly payment required by the credit card
  company each month.

  The following variables contain values as described below:
  \begin{itemize}

    \item \verb|balance| - the outstanding balance on the credit card
    \item \verb|annualInterestRate| - annual interest rate as a decimal
    \item \verb|monthlyPaymentRate| - minimum monthly payment rate as a decimal.

    \end{itemize}

  For each month, calculate statements on the monthly payment and remaining
  balance. At the end of 12 months, print out the remaining balance. Be sure to
  print out no more than two decimal digits of accuracy - so print

  \verb|Remaining balance: 813.41|

  instead of

  \verb|Remaining balance: 813.4141998135|

  So your program only prints out one thing: the remaining balance at the end of
  the year in the format:

  \verb|Remaining balance: 4784.0|

  A summary of the required math is found below:\\
  \textbf{Monthly interest rate}= (Annual interest rate) / 12.0
  \textbf{Minimum monthly payment} = (Minimum monthly payment rate) x (Previous balance)
  \textbf{Monthly unpaid balance} = (Previous balance) - (Minimum monthly payment)
  \textbf{Updated balance each month} = (Monthly unpaid balance) + (Monthly interest rate
  x Monthly unpaid balance)

  We provide sample test cases below. We suggest you develop your code on your
  own machine, and make sure your code passes the sample test cases, before you
  paste it into the box below.

  \subsection*{}
\end{document}